\begin{table}
\caption{Concepts and Measures of Regulatory Stringency}
\label{review}
\footnotesize
\centering
\raggedright 

\begin{tabular}{p{3.3cm}p{7.5cm}p{4.5cm}}
Selected Scholarship & Concept (arranged narrow to broad) &    Measurement orientation \\
\hline
\citet{Garcia-Montiel2017}&
“more rigorous,” “higher level,” “higher quality,” and thus “Greater complexity/Effectiveness/Cost” vs. “More Simplicity/Lower Cost”&
Number of indicators. Descriptions of consistency, coherence, and completeness.\\
\hline
\citet{Moore2012}&
Management practices changed &    
Survey of self-reported number and type of practices implemented\\
\hline
\citet{McDermott2010} &
“comprehensiveness and prescriptiveness”&
Number of key issues with most prescriptive language\\
\hline
\citet{Overdevest2014}&
“far apart” or “closer” on select “Characteristics”&
Binary table of select issues and descriptive examples\\
\hline
\citet{Overdevest2010}&
“comparative quality”--“weaker” are “revised upwards” to be “equivalent” to “higher and more prescriptive standards”&
Descriptive theory, examples, and review of previous comparisons\\
\hline
\citet{Fransen2011}&
“stringency” as “comprehensive in scope, specific in content, and prescriptive in terms of requirements”&
Description based on “leading policy analysts per issue area”\\
\hline
\citet{Hansen2006}&
Select “general features” and “six aspects” of management &
Descriptive table of select issues\\
\hline
\citet{Auld2014}&    
“Policy scope and regulatory domain,” “policy changes,” “character of the rules developed"&
Description of the set of problems addressed and how\\
\hline
\citet{Cashore2004}&    
“stringency”&
Descriptive theory, examples\\
\hline
\citet{Smith2010}&
 “stringency” of “weightings across multiple, and often conflicting, attributes,” also “excellence in content”&
Descriptive theory, examples\\
\hline
\citet{Porter2014}&
“hard law” or “soft law”&
Descriptive examples\\
\hline
\citet{Gulbrandsen2004}&
“variations in the strength”: “more stringent and less discretionary,” “more rigorous and wide-ranging” vs. “weak or lax” with “wider flexibility” Some “regulations have become more flexible” while others are “changing upward”&
Descriptive examples\\
\hline
\citet{Eberlein2014}&
“differentiation” along “dimensions of regulatory governance,” e.g. “more or less stringent” or “regulatory capacity”&
Descriptive typology, examples\\
\hline
\citet{Hassel2008}&
“high and low quality regulation,” “higher standards” vs. “lower standards”&
Descriptive theory, examples\\
\hline
\citet{Bartley2003}&
“more credible claims” vs. “lax standards” &
Descriptive theory, examples\\
\hline
\citet{Abbott2009}&
“regulatory outcomes”--“stringent” “higher standards” vs. “less stringent” “business-friendly” “weaker standards”&
Descriptive theory\\
\hline
\citet{Bernstein2007}&
Pressure to “raise” or “lower” requirements “explains convergence/ divergence”&
Descriptive theory\\
\hline
\citet{Kollman2001}, \citet{Potoski2005} &
“lax” or “processes-based” vs. “more stringent” “outcome-based” or “product-based” “types of regulations”&
ISO14001 classified as process-based, stringency assessed only for public regulations\\
\hline
\citet{Prakash2007}&
“stringent” vs. “lenient”&
Costs, social externalities, and branding benefits\\
\hline
Formal models of “stringency” or “quality”&
“sustainability quality level” \citep{Poret2016}, “stricter rules” \citep{Schmitz2017}, “stringency” \citep{Fischer2014}&
Costs vs. benefits to programs \& firms\\
\hline
Formal models of issue scope&
“issue-width” in an “issue space” (Hayes \& Martin 2015)&
Costs vs. benefits to programs and funders
\end{tabular}

\end{table}
