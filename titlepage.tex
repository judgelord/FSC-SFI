\documentclass[
      12pt,
        ]{article}






% --- type and typeface? -----------------------

% input
\usepackage[utf8]{inputenc}

% typography
\usepackage{microtype}


\usepackage[T1]{fontenc}


% text block
\usepackage{setspace}
\usepackage[              margin = 1in 
                        ]{geometry}

\usepackage{enumitem}
  \setlist{noitemsep}



% decimal numbering for appendix figs and tabs


% Deletes section counters
% \setcounter{secnumdepth}{0}







  \usepackage{longtable, booktabs}











% 

% \newtheorem{hypothesis}{Hypothesis}

\makeatletter
  \@ifpackageloaded{hyperref}{}{%
    \ifxetex
      % page size defined by xetex
      % unicode breaks when used with xetex
      \PassOptionsToPackage{hyphens}{url}\usepackage[setpagesize = false, 
                                                     unicode = false, 
                                                     xetex]{hyperref}
    \else
      \PassOptionsToPackage{hyphens}{url}\usepackage[unicode = true]{hyperref}
    \fi
  }

  \@ifpackageloaded{color}{
    \PassOptionsToPackage{usenames,dvipsnames}{color}
  }{
    \usepackage[usenames,dvipsnames]{color}
  }
\makeatother

\hypersetup{breaklinks = true,
            bookmarks = true,
            pdfauthor = {Devin Judge-Lord (University of Wisconsin--Madison) and Constance L. McDermott (Oxford University) and Benjamin Cashore (Yale University)},
             pdfkeywords  =  {policy change, private authority, certification, corporate social
responsibility, private governance},  
            pdftitle = {Do Private Regulations Ratchet Up?: How to distinguish types of regulatory stringency and patterns of change},
            colorlinks = true,
            citecolor = blue,
            urlcolor = blue,
            linkcolor = magenta,
            pdfborder = {0 0 0}}

% \urlstyle{same}  % don't use monospace font for urls


% set default figure placement to htbp
\makeatletter
  \def\fps@figure{hbtp}
\makeatother


% optional footnotes as endnotes


% ----- Pandoc wants this tightlist command ----------
\providecommand{\tightlist}{
  \setlength{\itemsep}{0pt}
  \setlength{\parskip}{0pt}
}





% --- title & section styles -----------------------


% title, author, date
  \title{Do Private Regulations Ratchet Up?: 
           \\ How to distinguish types of regulatory stringency and patterns of change}

  \author{ % author, option footnote, optional affiliation
            Devin Judge-Lord  \\ \emph{University of Wisconsin--Madison} 
             \and 
           % author, option footnote, optional affiliation
            Constance L. McDermott  \\ \emph{Oxford University} 
             \and 
           % author, option footnote, optional affiliation
            Benjamin Cashore  \\ \emph{Yale University} 
            }

% auto-format date?
  \date{\today}


% abstract
\usepackage{abstract}
  \renewcommand{\abstractname}{}    % clear the title
  \renewcommand{\absnamepos}{empty} % originally center

  \newcommand*{\authorfont}{\sffamily\selectfont}


% section titles
\usepackage[small, bf, sc]{titlesec}
  % \titleformat*{\subsection}{\itshape}
  \titleformat*{\subsubsection}{\itshape} 
  \titleformat*{\paragraph}{\itshape} 
  \titleformat*{\subparagraph}{\itshape}









\begin{document}
 

% --- PAGE: title and abstract -----------------------

  \maketitle

% \pagenumbering{gobble}



  \begin{abstract}
    \noindent Due to inconsistent measures of regulatory stringency, scholars offer
conflicting accounts about whether competing private governance
initiatives ``race to the bottom,'' ``ratchet up,'' ``converge,'' or
``diverge.'' To remedy this, we offer a framework to distinguish three
often-conflated measures: regulatory scope, prescriptiveness, and
performance levels. We use our framework to compare competing U.S.
forestry certification programs, one founded by environmental activists
and their allies, the other by the American Forest \& Paper Association.
We find ``upward'' but also divergent policy prescriptiveness, with the
activist-founded program adding requirements that impose costs on firms
and the industry-backed program mostly adding requirements with net
benefits to the sector. These results are consistent with the hypothesis
that industry-backed programs emphasize less costly types of stringency
than activist-backed programs. Furthermore, we find several more nuanced
patterns of change that previous scholarship failed to anticipate,
illustrating how disentangling types of stringency can improve theory
building and testing. 

          \hfill \\ 
      \noindent \emph{Keywords}: policy change, private authority, certification, corporate social
responsibility, private governance 
    
  \end{abstract}



% --- PAGE: contents -----------------------




% --- PAGE: body -----------------------



\noindent 
      \doublespacing 
    
% --- PAGE: endnotes -----------------------
% --- PAGE: refs -----------------------
\newpage
\singlespacing 
\end{document}
