\documentclass[]{article}
\usepackage{lmodern}
\usepackage{amssymb,amsmath}
\usepackage{ifxetex,ifluatex}
\usepackage{fixltx2e} % provides \textsubscript
\ifnum 0\ifxetex 1\fi\ifluatex 1\fi=0 % if pdftex
  \usepackage[T1]{fontenc}
  \usepackage[utf8]{inputenc}
\else % if luatex or xelatex
  \ifxetex
    \usepackage{mathspec}
  \else
    \usepackage{fontspec}
  \fi
  \defaultfontfeatures{Ligatures=TeX,Scale=MatchLowercase}
\fi
% use upquote if available, for straight quotes in verbatim environments
\IfFileExists{upquote.sty}{\usepackage{upquote}}{}
% use microtype if available
\IfFileExists{microtype.sty}{%
\usepackage{microtype}
\UseMicrotypeSet[protrusion]{basicmath} % disable protrusion for tt fonts
}{}
\usepackage[margin=1in]{geometry}
\usepackage{hyperref}
\hypersetup{unicode=true,
            pdfborder={0 0 0},
            breaklinks=true}
\urlstyle{same}  % don't use monospace font for urls
\usepackage{longtable,booktabs}
\usepackage{graphicx,grffile}
\makeatletter
\def\maxwidth{\ifdim\Gin@nat@width>\linewidth\linewidth\else\Gin@nat@width\fi}
\def\maxheight{\ifdim\Gin@nat@height>\textheight\textheight\else\Gin@nat@height\fi}
\makeatother
% Scale images if necessary, so that they will not overflow the page
% margins by default, and it is still possible to overwrite the defaults
% using explicit options in \includegraphics[width, height, ...]{}
\setkeys{Gin}{width=\maxwidth,height=\maxheight,keepaspectratio}
\IfFileExists{parskip.sty}{%
\usepackage{parskip}
}{% else
\setlength{\parindent}{0pt}
\setlength{\parskip}{6pt plus 2pt minus 1pt}
}
\setlength{\emergencystretch}{3em}  % prevent overfull lines
\providecommand{\tightlist}{%
  \setlength{\itemsep}{0pt}\setlength{\parskip}{0pt}}
\setcounter{secnumdepth}{0}
% Redefines (sub)paragraphs to behave more like sections
\ifx\paragraph\undefined\else
\let\oldparagraph\paragraph
\renewcommand{\paragraph}[1]{\oldparagraph{#1}\mbox{}}
\fi
\ifx\subparagraph\undefined\else
\let\oldsubparagraph\subparagraph
\renewcommand{\subparagraph}[1]{\oldsubparagraph{#1}\mbox{}}
\fi

%%% Use protect on footnotes to avoid problems with footnotes in titles
\let\rmarkdownfootnote\footnote%
\def\footnote{\protect\rmarkdownfootnote}

%%% Change title format to be more compact
\usepackage{titling}

% Create subtitle command for use in maketitle
\newcommand{\subtitle}[1]{
  \posttitle{
    \begin{center}\large#1\end{center}
    }
}

\setlength{\droptitle}{-2em}

  \title{}
    \pretitle{\vspace{\droptitle}}
  \posttitle{}
    \author{}
    \preauthor{}\postauthor{}
    \date{}
    \predate{}\postdate{}
  

\begin{document}

\begin{longtable}[]{@{}lll@{}}
\toprule
\begin{minipage}[b]{0.09\columnwidth}\raggedright\strut
\strut
\end{minipage} & \begin{minipage}[b]{0.41\columnwidth}\raggedright\strut
Program Level\strut
\end{minipage} & \begin{minipage}[b]{0.41\columnwidth}\raggedright\strut
Issue Level\strut
\end{minipage}\tabularnewline
\midrule
\endhead
\begin{minipage}[t]{0.09\columnwidth}\raggedright\strut
Policy Ends\strut
\end{minipage} & \begin{minipage}[t]{0.41\columnwidth}\raggedright\strut
How \textbf{comprehensive} is the \textbf{scope} of issues
addressed?\strut
\end{minipage} & \begin{minipage}[t]{0.41\columnwidth}\raggedright\strut
What are the \textbf{specific requirements (i.e.~policy settings)} on
each issues? (e.g.~the specific size of stream buffer zones)\strut
\end{minipage}\tabularnewline
\begin{minipage}[t]{0.09\columnwidth}\raggedright\strut
Policy Means\strut
\end{minipage} & \begin{minipage}[t]{0.41\columnwidth}\raggedright\strut
In aggregate, across all issues, how \textbf{prescriptive} is each
regulation? To what extent (e.g.~on what portion of issues) are there
mandatory and substantive thresholds?\strut
\end{minipage} & \begin{minipage}[t]{0.41\columnwidth}\raggedright\strut
1. How \textbf{prescriptive} is each requirement? 2. How are they
enforeced?*\strut
\end{minipage}\tabularnewline
\bottomrule
\end{longtable}

*Beyond the scope of this paper

\begin{figure}
\caption{}
\end{figure}

2010

\begin{longtable}[]{@{}lccc@{}}
\toprule
& Converging & Parallell & Diverging\tabularnewline
\midrule
\endhead
Increasing & 1 & 3 & 18\tabularnewline
Opposing or Eqilibrium & 0 & 21 & 3\tabularnewline
Decreasing & 2 & 0 & 0\tabularnewline
\bottomrule
\end{longtable}

2015

\begin{longtable}[]{@{}lccc@{}}
\toprule
& Converging & Parallell & Diverging\tabularnewline
\midrule
\endhead
Increasing & 3 & 0 & 0\tabularnewline
Opposing or Eqilibrium & 0 & 45 & 0\tabularnewline
Decreasing & 0 & 0 & 0\tabularnewline
\bottomrule
\end{longtable}


\end{document}
