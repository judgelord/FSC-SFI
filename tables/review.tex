\begin{table}
\caption{Concepts and measures of regulatory stringency}
\label{review}
\footnotesize

\begin{tabular}{p{3.5cm}p{8cm}p{5cm}}
Selected Scholarship & Concept &	Measurement orientation \\
\hline\\
García-Montiel et al. (2017)&
“more rigorous,” “higher level,” “higher quality,” and thus “Greater complexity/ Effectiveness/ Cost” vs. “More Simplicity/Lower Cost”&
Number of indicators. Descriptions of consistency, coherency, and completeness.\\
\hline\\
Moore et al. (2012)&
Management practices changed &	
Survey of self-reported number and type of management practices implemented\\
\hline\\
McDermott et al. (2010; 2008)&
“comprehensiveness and prescriptiveness”&
Number of key issues with most prescriptive language\\
\hline\\
Overdevest \& Zeitlin (2012)&
“far apart” or “closer” on select “Program Characteristics”&
Binary table of select issues and descriptive examples\\
\hline\\
Overdevest (2005, 2010)&
“comparative quality”— “weaker” standards are “revised upwards” to be “equivalent” to “higher and more prescriptive standards”&
Descriptive theory, examples, and review of previous comparisons\\
\hline\\
Fransen (2011), Fransen \& Conzellman (2015)&
“stringency” as “comprehensive in scope, specific in content, and prescriptive in terms of requirements”&
Descriptions based on “leading policy analysts per issue area”\\
\hline\\
Hansen et al. (2006)&
Select “general features” and “six aspects” of management &
Descriptive table of select issues\\
\hline\\
Auld (2014)&	
“Policy scope and regulatory domain” “policy changes,” “character of the rules developed"&
Description of the set of problems addressed and how\\
\hline\\
Cashore et al. (2004)&	
“stringency”&
Descriptive theory and examples\\
\hline\\
Smith \& Fischlein (2010)&
 “stringency” of “weightings across multiple, and often conflicting, attributes,” also “excellence in content”&
Descriptive theory and examples\\
\hline\\
Porter (2014)&
“hard law” or “soft law”&
Descriptive examples\\
\hline\\
Gulbrandsen (2004)&
“variations in the strength of standards”—“more stringent and less discretionary,” “more rigorous and wide-ranging” vs. “weak or lax” “and allow far wider flexibility”
Some “regulations have become more flexible” while others are “changing upward”&
Descriptive examples\\
\hline\\
Eberlein et al. (2014)&
“differentiation among rule systems” along “dimensions of regulatory governance,” e.g. “more or less stringent requirements” or “regulatory capacity”&
Descriptive typology and examples\\
\hline\\
Hassel (2008)&
“high and low quality regulation”, “higher standards” vs. “lower standards”&
Descriptive theory and examples\\
\hline\\
Bartley (2003)&
“more credible claims” vs. “lax standards” &
Descriptive theory and examples\\
\hline\\
Abbott \& Snidal (2008)&
“substance and form of regulatory outcomes”—
 “stringent” “higher standards” vs. “less stringent” “business-friendly” “weaker standards”&
Descriptive theory\\
\hline\\
Bernstein \& Cashore (2007)&
Pressure to “raise” or “lower” requirements “explains convergence/ divergence”&
Descriptive theory\\
\hline\\
Kollman \& Prakash (2011), Potoski \& Prakash (2004), Prakash \& Potoski 2006)&
“lax” or “processes-based” vs. “more stringent” “outcome-based” or “product-based” “types of regulations”&
ISO14001 classified as process-based, stringency assessed only for public regulations\\
\hline\\
Prakash \& Potoski (2007)&
“stringent” vs. “lenient” standards&
Proportional costs, social externalities, and branding benefits\\
\hline\\
Formal models of “stringency” or “quality”&
“sustainability quality level” (Poret 2016), “more ambitious” (Fischer et al., 2017), “stricter rules” (Schmitz et al., 2017), “stringency” (Fischer \& Lyon 2014; Hayes \& Martin, 2017)&
Proportional costs \& benefits to programs \& firms\\
\hline\\
Formal models of issue scope&
“issue-width” in an “issue space” (Hayes \& Martin 2015)&
Proportional costs and benefits to programs and funders
\end{tabular}

\end{table}
